\documentclass[norsk,a4paper]{article}
\usepackage[norsk]{babel}
\usepackage[utf8]{inputenc}
\usepackage{parskip,textcomp,fullpage,fancyhdr,graphicx}
\usepackage[bitstream-charter]{mathdesign}
\usepackage[T1]{fontenc}

\pagestyle{fancy}
\fancyhf{}
\renewcommand{\headrulewidth}{0pt}

\lfoot{\thepage}
\rfoot{\includegraphics[width=3cm,trim=0 4cm 0 0]{../res/logo.png}}

% Redefines sections and subsections in the format §1-1
\renewcommand{\thesection}{§\arabic{section}}
\renewcommand{\thesubsection}{\thesection-\arabic{subsection}}

% Redefines first level of enumerate in the format a), b), ...
\renewcommand{\theenumi}{\alph{enumi}}
\renewcommand{\labelenumi}{\theenumi)}
\renewcommand{\labelenumii}{\alph{enumi}.\roman{enumii}}

\title{\textbf{Vedtekter for Ifi-dagen} \\
{\large Oppdatert etter generalforsamlingen 27.\ oktober 2021}}
\date{}

\begin{document}

\maketitle{}
\thispagestyle{fancy}

\section{Formål}
\subsection{Foreningens formål}
\begin{enumerate}
    \item{Ifi-dagen er en studentforening ved Institutt for informatikk (Ifi) ved Universitetet i Oslo (UiO).}
    \item{Ifi-dagen skal jobbe for å skape kontakt mellom studenter ved Ifi og næringslivet, dette gjøres hovedsakelig gjennom hver høst å arrangere karrieredagen dagen@ifi.}
\end{enumerate}
\subsection{dagen@ifi}
\begin{enumerate}
    \item{dagen@ifi skal være et arrangement for studentene ved Ifi. Arrangementet skal strebe etter å engasjere studentene ved instituttet, profilere studentene og informatikkmiljøet ved UiO overfor næringslivet og bidra til det sosiale miljøet på Ifi. Arrangementet skal også gjennom et bredt faglig program vise informatikkstudentene interessante sider av informatikkfaget.}
    \item{Styret skal innen 15.\ april annonsere dato for årets dagen@ifi.}
\end{enumerate}

\section{Foreningens ledelse og forvaltning}
\subsection{Generalforsamlingen og dens myndighet}
\begin{enumerate}
    \item{Ifi-dagen sin høyeste myndighet er dets generalforsamling. Ordinær generalforsamling avholdes årlig, etter avholdt dagen@ifi og før utgangen av det gjeldende året. Ekstraordinær generalforsamling avholdes når styret, eller minst 30 registrerte studenter ved Ifi ønsker det.}
    \item{Alle studenter ved Ifi er stemmeberettigede på generalforsamling.}
    \item{Styret innkaller til ordinær generalforsamling med minst to ukers varsel. For ekstraordinær generalforsamling minst en ukes varsel.}
    \item{Foreløpig dagsorden offentliggjøres minst en uke i forkant av ordinær generalforsamling. For ekstraordinær generalforsamling minst tre dager i forkant.}
    \item{For at et vedtak på generalforsamlingen skal være gyldig, må det være minst 20 stemmeberettigede tilstede.}
    \item{Ved endringer av foreningens vedtekter kreves det 2/3 flertall blant de tilstedeværende stemmeberettigede.}
    \item{Forslag til vedtektsendringer skal være styret i hende senest 48 timer før generalforsamling.}
\end{enumerate}
\subsection{Valg av styre}
\begin{enumerate}
    \item{Styret velges av generalforsamlingen, og skal bestå av studenter ved Ifi.}
    \item{Funksjonstiden for et styremedlem varer enten frem til neste ordinære generalforsamling, eller fram til utgangen av året som styret ble innvalgt for, hva enn som er lengst.}
    \item{Styret skal minimum bestå av leder, økonomiansvarlig og bedriftsansvarlig, og disse vervene kan kun velges på en generalforsamling.}
    \item{Styret kan velge andre verv til valg før generalforsamlingen.}
    \item{Dersom verv fra punkt d ikke er fylt etter generalforsamling, eller et styremedlem fratrer, kan styret selv etterfylle nye styremedlemmer som blir sittende til neste ordinære generalforsamling.}
    \item{Ved frafall av et styremedlem (utenom leder og økonomiansvarlig), kan styret selv etterfylle nye styremedlemmer som blir sittende til neste ordinære generalforsamling. Leder og økonomiansvarlig må velges på ekstraordinær generalforsamling.}
    \item{Valg av styremedlemmer på generalforsamling skjer ved Instant-runoff vote, også kjent som IRV, eller alternativ stemmegivning.}
\end{enumerate}
\subsection{Styret og dets oppgaver}
\begin{enumerate}
    \item{Styret er ansvarlig for Ifi-dagen sin virksomhet og for at foreningens vedtekter følges.}
    \item{Styret er beslutningsdyktig dersom minst 2/3 av styremedlemmene er tilstede.}
    \item{Styret avgjør saker med alminnelig flertall ved stemmegivning. Ved stemmelikhet har styreleder to stemmer.}
    \item{Styret kan gjøre redaksjonelle endringer på vedtektene.}
\end{enumerate}

\section{Signaturrett}
\subsection{Signering på vegne av Ifi-dagen}
\begin{enumerate}
    \item{Styreleder og ett styremedlem kan, i samsvar med hverandre, signere på vegne av Ifi-dagen.}
    \item{De øvrige vervene kan signere på vegne av styret i saker som omhandler deres verv, i samråd med styret.}
\end{enumerate}

\section{Økonomi\label{sec:okonomi}}
\subsection{Fordeling av overskudd}
\begin{enumerate}
    \item{Ved innføring av nytt styre skal et eventuelt overskudd overføres til Fordelingsutvalget ved Institutt for informatikk, foruten en egenkapital for foreningen på kr 200 000,--.}
    \item{Fordelte midler fra et eventuelt overskudd skal gå til formål som kommer studentene og deres studentforeninger ved Institutt for informatikk til gode.}
\end{enumerate}

\subsection{Informasjonsplikt}
\begin{enumerate}
    \item{Ifi-dagen plikter å holde Fordelingsutvalget ved Ifi oppdatert med relevant regnskapsinformasjon ved forespørsel og regnskapsavslutning.}
    \item{Ved forslag om endring av~\ref{sec:okonomi} skal Fordelingsutvalget ved Ifi informeres skriftlig før generalforsamling.}
\end{enumerate}

\section{Interne}
\begin{enumerate}
    \item{Interne bistår styret med å planlegge og avholde arrangementene.}
    \item{Ifi-studenter kvalifiserer til å bli intern i Ifi-dagen og har dermed rett til å søke Ifi-dagens styre om å bli intern.}
    \item{Interne godkjenens av styret.}
    \item{Interne velges for perioden styret sitter, og vil få mulighet til å fortsette som intern når nytt styre har blitt valgt.}
    \item{Interne kan ved skriftelig beskjed til styre si fra seg sin intern stilling med umiddelbar virkning.}
\end{enumerate}

\section{Alumni}
\begin{enumerate}
    \item{Medlemmer i Ifi-dagen alumni er tidligere styremedlemmer. Ifi-dagen styre kan oppnevne interne som har utmerket seg.}
    \item{Ifi-dagen alumno er en selvorganisert undergruppe.}
    \item{Medlemmene i Ifi-dagen alumni har møte- og talerett på Ifi-dagens generalforsamling.}
\end{enumerate}

\section{Mistillit}
Mistillitsforslag overfor foreningens styremedlemmer skal behandles på generalforsamling, og krever 2/3 flertall for å bli vedtatt. Mistillitsforslag må fremmes senest 48 timer før generalforsamling.

\section{Oppløsning}
\subsection{Oppløsning av Ifi-dagen}
        For å gjennomføre foreslått oppløsning av Ifi-dagen som organisasjon, må det være 9/10 flertall
        blant de fremmøtte stemmeberettigede på 2 generalforsamlinger.
\subsection{Fordeling av midler ved oppløsning}
\begin{enumerate}
        \item{Ved oppløsning skal Ifi-dagen sine økonomiske midler og eiendommer overføres til Fordelingsutvalget ved Ifi.}
        \begin{enumerate}
            \item Styret plikter til å ha utarbeidet en investeringsplan for foreningens fond, i samarbeid med en økonomisk rådgiver. Investeringsplan skal evalueres i forkant av hver generalforsamling, og presenters for studentene.
            \item Fondets formål skal være å sikre den økonomiske fremtiden til studentforeninger ved Institutt for Informatikk.
        \end{enumerate}
        \item{Fordelte midler fra et eventuelt overskudd skal gå til formål som kommer studentene og deres studentforeninger ved Institutt for informatikk til gode.}
\end{enumerate}

\end{document}
